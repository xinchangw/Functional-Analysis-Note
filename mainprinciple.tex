\section{Main Principles of Functional Analysis}
\subsection{Bounded Linear Operators}

\begin{definition}\ \\
    $X,Y$ Banach spaces, we say $T: X \to Y$ is a \textit{linear operator} if 
    \begin{equation*}
        T(ax + by) = a Tx + b Ty,\ a,b \in \R \text{ or }\C,\ x,y \in X.
    \end{equation*}
    The operator $T$ is said to be \textit{bounded} if
    \begin{equation*}
        \norm{T} := \sup_{\norm{x} = 1} \norm{Tx} < \infty,
    \end{equation*}
    then $T$ is \textit{Lipschitz}:
    \begin{equation*}
        \norm{Tx - Ty} \leq \norm{T} \norm{x - y},\ x,y \in X.
    \end{equation*}
\end{definition}

\begin{remark}\ \\
The operator norm is a norm on bounded linear operators.
\end{remark}

\vspace{6pt}
\begin{proposition}\ \\
Let $\Ls(X,Y) = \text{ space of bounded linear operators } T: X \to Y$, then $\Ls(X,Y)$ is a \textit{Banach space} with the norm $T \to \norm{T}$, i.e.
\begin{enumerate}
    \item $\norm{T} = 0 \Leftrightarrow T = 0$.
    \item $\norm{\lambda T } = \abs{\lambda} \norm{T},\ \lambda \in \R \text{ or } \C,\ T \in \Ls(X,Y)$.
    \item $\norm{T + S} \leq \norm{T} + \norm{S}$, $T,S \in \Ls(X,Y)$.
    \item In addition operator norm satisfies
    \begin{equation*}
        \norm{TS} \leq \norm{T} \norm{S},\ T,S \in \Ls(X,Y).
    \end{equation*}
\end{enumerate}
\end{proposition}

\vspace{6pt}
\begin{definition}[Adjoint operator]\ \\
Suppose $T \in \Ls(X,Y)$, then the adjoint operator $T^*$ of $T$ is defined as (for $f \in Y^*$)
\begin{equation*}
    T^* f : X \to \R \text{ or } \C,\ T^*f(x) = f(Tx),
\end{equation*}
and we further have
\begin{itemize}
    \item $\abs{T^* f(x)} = \abs{f(Tx)} \leq \norm{f} \norm{Tx} \leq \norm{f} \norm{T} \norm{x}$
    \item $T^* f$ is a linear functional.
    \item $\norm{T^* f} = \sup_{\norm{x} = 1} \norm{f}\norm{T}\norm{x} = \norm{f} \norm{T}$.
\end{itemize}
Hence $T^* f \in X^*$, and 
\begin{equation*}
    \norm{T^* f} \leq \norm{T}\norm{f},
\end{equation*}
and thus $T^*: Y^* \to X^*$ is a linear operator and bounded with $\norm{T^*} \leq \norm{T}$.
\end{definition}


\vspace{3pt}
\begin{proposition}\ \\
For every $T \in \Ls(X,Y)$, the adjoint $T^*$ is in $\Ls(Y^*, X^*)$ and $\norm{T^*} = \norm{T}.$
\end{proposition}
\begin{proof}\ 
\begin{align*}
    \norm{T^*} &= \sup_{\norm{f}_{Y^*} = 1} \norm{T^* f}_{X^*}\\ 
    &= \sup_{\norm{f}_{Y^*} = 1} \sup_{\norm{x}_X = 1} \abs{T^* f(x)}\\
    &= \sup_{\norm{f}_{Y^*} = 1} \sup_{\norm{x}_X = 1} \abs{f(Tx)}\\
    &=  \sup_{\norm{x}_X = 1} \sup_{\norm{f}_{Y^*} = 1}\abs{f(Tx)}.
\end{align*}
From \hyperref[SHT]{Supporting Hyperplane Theorem}, 
\begin{equation*}
    \sup_{\norm{f}_{Y^*} = 1} \abs{f(Tx)} = \norm{Tx},
\end{equation*}
conclude that
\begin{equation*}
    \norm{T^*} = \sup_{\norm{x}_X = 1} \norm{Tx} = \norm{T}.
\end{equation*}
\end{proof}

\vspace{3pt}
\begin{proposition}[Properties of Adjoint Operators]\ \\
For $T,S \in \Ls(X,Y)$, we have $T^*,S^* \in \Ls(Y^*, X^*)$, then
\begin{enumerate}[label = (\arabic*)]
    \item $(aT + bS)^* = aT^* + b S^*,\ a,b \in \R\ \&\ \C.$
    \item $(a T)^* f(x) = f(a Tx) = af(Tx) = aT^* f(x)$.
    \item $(ST)^* = T^* S^*$.
\end{enumerate}
\end{proposition}
\begin{remark}\ \\
From (2), if $T \in \Ls(X,X)$ is invertible, then $T^* \in \Ls(X^*,X^*)$ is invertible and $(T^*)^{-1} = (T^{-1})^*$.
\end{remark}

\begin{remark}\ \\
Just as Hilbert space, we have the notion of orthogonality.
\end{remark}

\vspace{3pt}
\begin{definition}[Annihilator]\ \\
Let $A \subset X$ be a Banach space, the \textit{Annihilator} $A^\perp$ of $A$ is a subset of $X^*$ s.t.
\begin{equation*}
    A^\perp = \{f \in X^*:\ f(x) = 0,\ \forall\ x \in A\}.
\end{equation*}
$A^\perp$ is a closed linear subspace of $X^*$.
\end{definition}
\begin{remark}\ \\
Just as in Hilbert space, \textit{Annihilator} is a notion indicating \textit{orthogonality}. 
\end{remark}

\vspace{3pt}
\begin{proposition}\label{5.7}\ \\
Let $T \in \Ls(X,Y)$, $T^* \in \Ls(Y^*, X^*)$, where $X,Y$ are Banach spaces, then
\begin{center}
    $(\Im{T})^\perp \subset Y^*$ and $\ker{T^*} \subset Y^*$
\end{center}
satisfy $(\Im{T})^* = \ker{T^*}$.
\end{proposition}
\begin{proof}\ \\
$f \in \Im{T}^\perp \Leftrightarrow f(Tx) = 0,\ \forall\ x \in X$, i.e.  $T^*f(x) = 0,\ \forall\ x \in X \Leftrightarrow T^*f = 0 \Leftrightarrow f \in \ker{T^*}$.
\end{proof}

\vspace{3pt}
\begin{remark}[Adjoint Operators on Hilbert Space]\ \\
Suppose $\Hs = \text{ Hilbert space }$, then by \hyperref[RRT]{Riesz Representation Theorem}
\begin{equation*}
    \Hs^* = \Hs,
\end{equation*}
i.e. $f \in \Hs^* \Leftrightarrow\ \exists\ y \in \Hs,\ f(x) = \inprod{x}{y},\ x \in \Hs$.

\np Let $T \in \Ls(\Hs,\Hs),\ T^* \in \Ls(\Hs^*,\Hs^*)$, we have
\begin{equation*}
    T^*f(x) = f(Tx) = \inprod{Tx}{y},\ x \in \Hs,\ f \in \Hs^*.
\end{equation*}

\np Write $T^* f(x) = \inprod{x}{T^*y}$ by \hyperref[RRT]{Riesz Representation Theorem}, which defines $T^*y$ and $T^*: \Hs \to \Hs$.

\np Hence $\inprod{Tx}{y} = \inprod{x}{T^* y},\ x,y \in \Hs$. Clearly, $T^*$ is a bounded linear operator on $\Hs$, i.e, $T^* \in \Ls(\Hs, \Hs)$, and we have $\norm{T^*} = \norm{T}$, since
\begin{equation*}
    \norm{T^*} = \sup_{\norm{y} = 1} \norm{T^* y} = \sup_{\norm{y} = 1,\ \norm{x} = 1} \inprod{x}{T^* y} = \sup_{\norm{y} = 1,\ \norm{x} = 1} \inprod{Tx}{y} = \norm{T}.
\end{equation*}
Therefore, $T^* \in \Ls(\Hs^*,\Hs^*) \Rightarrow T^* \in \Ls(\Hs, \Hs)$.

\np Note that if $T^* \in \Ls(\Hs,\Hs)$, we have
\begin{equation*}
    (aT)^* = \bar{a} T^*,\ a \in \C,\ \bar{a} = \text{ complex conjugate }.
\end{equation*}
\end{remark}


\vspace{6pt}
\begin{corollary}\ \\
    Let $\Hs$ be a Hilbert space, and $T \in \Ls(\Hs,\Hs)$. Then we have the orthogonal decomposition   
    \begin{equation*}
         \bar{\Im{T}} \bigoplus \ker(T^*) = \Hs.
    \end{equation*}
\end{corollary}
\begin{proof}\ \\
Note that $T \in \Ls(X,Y) \Rightarrow T^* \in \Ls(Y^*, X^*)$, and \hyperref[5.7]{Proposition 5.7} implies $\Im{T}^\perp = \ker{T^*}$. We use the fact that since $E$ is a subspace of $\Hs$, then 
\begin{equation*}
     (E^\perp)^\perp = \bar{E} = \text{ closure of } E.
\end{equation*}
Therefore, $\ker{T^*} = (\bar{\Im{T}})^\perp$. Complete the proof.
\end{proof}


\vspace{12pt}
\subsubsection{Ergodic Theory}

\np We can apply the above results to prove \textit{Ergodic Theory}, which allows one to compute space average via time average.

\begin{definition}\ \\   
Given probability space $(\Omega,\ \Fs,\ \mathbb{P})$, let $T: \Omega \to \Omega$ be a measurable map, we say $T$ is \textit{measure preserving} if
\begin{equation*}
    \P(T^{-1} A) = \P(A),\ A \in \Fs.
\end{equation*}
where $T^{-1}A = \{\omega \in \Omega:\ T \omega \in A\}$. 
\end{definition}


\begin{example}[Rotation]\ \\
$\Omega = [0,1]$, $\P = $Lebesgue measure, $\Fs = \text{ Borel } \sigma-\text{algebra}$. Given $\lambda \in \R$, let
\begin{equation*}
    T \omega = \omega + (\lambda \text{ mod } 1).
\end{equation*}
This is equivalent to rotation on the unit sphere through an angle $2\pi \lambda$. $T$ is measure preserving and one-to-one correspondence. $T^{-1}$ exists.
\end{example}

\begin{example}[Shift Operator]\ \\
$\Omega = [0,1]$, $\P = \text{ Lebesgue measure },\ \Fs = \text{ Borel sets}$. Let
\begin{equation*}
    T \omega = 2 \omega \text{ mod } 1.
\end{equation*}
$T$ is the shift operator on the binary representation, i.e. $\omega = \summ{j=1}\infty \frac{a_j}{2^j},\ a_j = 0 \text{ or } 1$. Therefore,
\begin{equation*}
    T \omega = \summ{j = 1}\infty \frac{a_{j + 1}}{2^j}.
\end{equation*}
Let
\begin{equation*}
    I_{n,k} = [\frac{k - 1}{2^n}, \frac{k}{2^n}],\ 1 \leq k < 2^n,\ n = 1,2,\dots
\end{equation*}
which are the Dyadic intervals. We have
\begin{equation*}
    T^{-1} I_{n,k} = I_{n+1,k} \cup I_{n+1, k+ 2^n}.
\end{equation*}
We have $\P(T^{-1} I_{n,k}) = \P(I_{n,k})$ for all Dyadic interval, which implies
\begin{equation*}
    \P(T^{-1} O) = P(O),\ \forall\ O \in \Fs.
\end{equation*}
Hence $T$ is measure preserving but $T$ is \underline{not} one-to-one correspondence. In fact, $T$ is a $2 \to 1$ mapping. Action of $T$ is
\begin{align*}
    [0,\frac{1}{2}] &\overset{T}{\to} [0,1],\\
    [\frac{1}{2},1] &\overset{T}{\to} [0,1].
\end{align*}
$T$ doubles the length of a dyadic interval, and it is measure preserving since it is two-to-one. $T$ is an expanding map, such mappings are called \textit{hyperbolic}.
\end{example}


\np Suppose $T: \Omega \to \Omega$ is measure-preserving, we can associate operator $U$ on $\Ls^2(\Omega)$ by $U f(\omega) = f(T \omega)$, where $f \in \Ls^2(\Omega),\ \omega \in \Omega$. Then we have
\begin{equation*}
    \int_\Omega f(T \omega) \dr \mu(\omega) = \int_\Omega f(\omega) \dr \mu(\omega),\ \forall\ f \in \Ls^1(\Omega)
\end{equation*}
It is true if $f = \1_A$ and then extend to $\Ls^1(\Omega)$.\\
Suppose $\vphi \in \Ls^2(\Omega)$, we have $U \vphi(\omega) = \vphi (T \omega)$. Also,
\begin{align*}
    \inprod{U \vphi}{U \psi} &= \int_\Omega \vphi(T \omega) \psi(T \omega) \dr \mu(\omega)\\
    &= \int_\Omega \vphi(\omega) \psi(\omega) \dr \mu(\omega)\\
    &= \inprod{\vphi}{\psi}.
\end{align*}
Conclude that
\begin{equation*}
    \inprod{U \vphi}{U \psi} = \inprod{\vphi}{\psi},\ \vphi,\psi \in \Ls^2(\Omega).
\end{equation*}

\np $U$ is a bounded linear operator on $\Hs = \Ls^2(\Omega)$ with $\norm{U} = 1,\ \norm{U\vphi} = \norm{\vphi},\ \vphi \in \Hs$.\\
In addition,
\begin{equation*}
    \inprod{U \vphi}{U \psi} = \inprod{\vphi}{\psi} \Rightarrow \inprod{U^* U \vphi}{\psi} = \inprod{\vphi}{\psi},\ \vphi,\psi \in \Hs,
\end{equation*}

\np which implies $U^*U = I$. Therefore, $U$ is one-to-one but not necessarily onto.

\begin{definition}[Unitary Operator]\ \\
A \textit{unitary operator} on a Hilbert space $\Hs$ is a bounded linear operator $U: \Hs \to \Hs$ such that $U$ is surjective and $\forall\ x,y \in \Hs$,
\begin{equation*}
    \inprod{Ux}{Uy}_{\Hs} = \inprod{x}{y}_{\Hs}.
\end{equation*}

\end{definition}
\begin{remark}\ \\
$U$ is invertible if.f $T$ is onto.
\begin{proof}\ \\
If $U$ is onto, then $U^* U = U U^* = I$, and thus $U$ is invertible. Note $U^* \vphi(\omega) = \vphi(T^{-1} \omega),\ \omega \in \Omega$. If $T$ is one-to-one, then $T^{-1}$ is onto, implying $U^*$ is onto, so is $U$.
\end{proof}
\end{remark}

\begin{remark}\ \\
$T$ (1-1) implies $T$ almost onto. To see this let $A$ be a set s.t. $T(\Omega) \subset A$, and hence $T^{-1} A = \Omega,\ \P(T^{-1}A) = \P(\Omega) = \Delta$ $\Rightarrow$ $\P(A) = 1$ $\Rightarrow$ $\P(\Omega \setminus A) = 0$. 
\end{remark}

\np In the case $T$ is not invertible (e.g. 2-1 mapping), might expect similar formula for $U^*$. In the shift operator case, $T_1:[0,\frac{1}{2}] \to [0,1],\ T_2: [\frac{1}{2},1] \to [0,1]$, and $T_1, T_2$ are invertible, we have
\begin{equation*}
    U^* \vphi(\omega) = \frac{1}{2}[\vphi(T_1^{-1} \omega) + \vphi(T_2^{-1} \omega)].
\end{equation*}

\vspace{3pt}
\begin{definition}\ \\
A measure preserving mapping $T: \Omega \to \Omega$ is \textit{ergodic} provided the only eigen function $\vphi \in \Ls^2(\Omega)$ of the corresponding operator $U$ is the constant function, i.e.
\begin{equation*}
    U \vphi = \vphi \Rightarrow \vphi \equiv \text{constant}.
\end{equation*}
\end{definition}

\vspace{3pt}
\begin{lemma}\ \\
A measure preserving mapping $T: \Omega \to \Omega$ is \textit{ergodic} \underline{if and only if} invariant sets of $T$ have probability in $\{0,1\}$, i.e. if $A \in \Fs$ satisfies 
\begin{equation*}
    \P[(A - T^{-1} A) \cup (T^{-1} A - A)] = 0, 
\end{equation*}
then $\P(A) = 0$ or $\P(A) = 1$.
\end{lemma}
\begin{proof}\ \\
Assume $T$ is not ergodic, then $\exists\ \vphi \in \Ls^2(\Omega),\ U \vphi = \vphi$. Hence can find $a,b \in \R,\ a < b$ s.t. 
\begin{equation*}
    A = \{\omega \in \Omega:\ a < \vphi(\omega) < b\}
\end{equation*}
has $0 < \P(A) < 1$. However,
\begin{align*}
    T^{-1}A &= \{\omega:\ T \omega \in A\}\\
    &= \{\omega:\ a < \vphi(T \omega) < b\}\\
    &= \{\omega:\ a < \vphi(\omega) < b\} = A,
\end{align*}
and thus $A$ is invariant.

\np Conversely suppose $A \in \Fs$, we have $A = T^{-1}A$ up to measure-zero sets and $0 < \P(A) < 1$, then $\vphi = \1_A$ satisfies $U \vphi = \vphi,\ \vphi \in \Ls^2(\Omega),\ \vphi \not\equiv \text{constant}$.
\end{proof}


\begin{theorem}[Von Neumann Ergodic Theorem, v.1] \label{VN ergodic}\ \\
Suppose $T: \Omega \to \Omega$ is measure preserving, then for any $\vphi \in \Ls^2(\Omega)$, one has 
\begin{equation*}
    \lim_{N \to \infty} \frac{1}{N} \summ{n=0}{N-1} \vphi(T^n \cdot) = \int_\Omega \vphi(\omega) \dr \P(\omega).
\end{equation*}
\end{theorem}
\begin{remark}\ \\
Convergence is in the $\Ls^2(\Omega)$ sense, i.e. mean square.
\end{remark}


\begin{proposition}[Von Neumann Ergodic Theorem, v.2]\ \\
$T: \Omega \to \Omega$ measure preserving, $\vphi \in \Ls^2(\Omega),\ \E[\vphi] = 0$, then
\begin{equation*}
    \frac{1}{N} \summ{n = 0}{N - 1} \vphi(T^n \cdot) \to 0
\end{equation*}
in $\Ls^2(\Omega)$ as $N \to \infty$. 
\end{proposition}
\begin{proof}\ \\
Note it suffices to assume $\E[\vphi] = 0$. WTS 
\begin{equation*}
     \frac{1}{N} [I + U + U^2 + \dots + U^{N-1}] \vphi(\cdot) \to 0 \text{ in } \Ls^2(\Omega).
\end{equation*}
Note $\vphi$ is orthogonal to the constant function, since $\E[\vphi] = 0$ which implies $\inprod{\vphi}{1} = 0$. Define a "derivative" operator on $\Ls^2(\Omega)$ s.t.
\begin{equation*}
    D \vphi = (U - I) \vphi = \vphi(T \cdot) - \vphi(\cdot).
\end{equation*}
Use Fundamental Theorem of Calculus argument,
\begin{equation*}
    [I + U + U^2 + \dots + U^{N-1}] D\vphi = (U^N - I) \vphi.
\end{equation*}
Hence,
\begin{equation*}
    \norm{\frac{I + U + U^2 + \dots + U^{N-1}}{N} \vphi} \leq \frac{2 \norm{\psi}}{N}
\end{equation*}
if $\vphi = D \psi$. In that case $\lim$ as $N \to \infty$ is zero, i.e. if $\vphi \in \Im{D} \subset \Hs = \Ls^2(\Omega)$, then finished. Note also that
\begin{equation*}
    \norm{\frac{I + U + U^2 + \dots + U^{N-1}}{N}} \leq 1
\end{equation*}
since $\norm{U} = 1$. Hence converge to zero if $\vphi \in \bar{\Im{D}} = \text{ closure of } \Im{D}$. 
\begin{equation*}
    \vphi \in \bar{\Im{D}} \Rightarrow \exists\ \vphi_\ep \in \Im{D},\ \norm{\vphi_\ep - \vphi} < \ep,
\end{equation*}
which implies $\norm{\frac{I + U + \dots + U^{N-1}}{N} (\vphi_\ep - \vphi)} < \ep$.

\np Recall $\bar{\Im{D}} \bigoplus \ker{D^*} = \Hs = \Ls^2(\Omega)$. It suffices to show $\ker{D^*}$ is spanned by constant functions. 

\np Note $T$ ergodic implies $\ker{D}$ is spanned by constants, we have $D \vphi = 0 \Leftrightarrow U\vphi = \vphi$, and
\begin{equation*}
    (D^* \vphi = 0 \Leftrightarrow U^* \vphi = 0) \Rightarrow (\inprod{\vphi, U^* \vphi}{\vphi} = \inprod{\vphi}{\vphi}). 
\end{equation*}
Therefore,
\begin{align*}
    \inprod{U \vphi}{\vphi} &= \inprod{\vphi}{\vphi}\\
    \int \vphi(T \omega) \vphi(\omega) \dr \P(\omega) &=\int \vphi(\omega)^2 \dr \omega\\
    &= \int \vphi(T \omega)^2 \dr \omega,
\end{align*}
which implies
\begin{equation*}
    \frac{1}{2} \int [\vphi(T \omega)^2 + \vphi(\omega)^2] \dr \P(\omega) - \int \vphi(T \omega) \vphi(\omega) \dr \P(\omega) = 0.
\end{equation*}
i.e. $\frac{1}{2} \int[\vphi(T \omega) - \vphi(\omega)]^2 \dr \P(\omega) = 0$, which means
\begin{equation*}
    \vphi(T \omega) = \vphi(\omega),\ \omega \in \Omega.
\end{equation*}
i.e. $\vphi \equiv \text{constant}$ by ergodicity.
\end{proof}

\vspace{12pt}
\subsection{Open Mapping Theorem}

\np Suppose $T: X \to Y$ bounded linear operator on Banach spaces, and $T$ is injective and surjective, i.e. $T^{-1}: Y \to X$ exists. \underline{Open Mapping Theorem} implies $T^{-1}$ is a bounded operator. The main input into argument is \underline{Baire Category Theorem}. 

\vspace{3pt}
\begin{definition}\ \\
A set $S$ in a metric space $M$ is \underline{nowhere dense} (e.g. Cantor set) if its closure $\bar{S}$ has empty interior. 
\end{definition}

\vspace{3pt}
\begin{proposition}[Baire Category]\label{BCT}\ \\
A complete metric space is \underline{never} the union of a countable number of nowhere dense sets.
\end{proposition}
\begin{proof}\ \\
Argue by contradiction. Assume $M = \cupp{n=1}{\infty} A_n$ with each $A_n$ nowhere dense.

\np $A_1$ is nowhere dense $\Rightarrow$ can find $x_1 \in M - \bar{A_1}$.

\np $\bar{A_1}$ closed $\Rightarrow$ can find open ball $B_1$ centered at $x_1$ with radius $\leq 1$ s.t. $B_1 \cap A_1 = \emp$.

\np $A_2$ is nowhere dense $\Rightarrow$ $\exists\ x_2 \in B_1 - \bar{A_2}$.

\np $\bar{A_2}$ closed so can find ball $B_2$ centered at $x_2$ with radius $\leq \frac{1}{2}$ s.t. $x_2 \in B_2 \subset \bar{B_2} \subset B_1$ and $B_2 \cap A_2 = \emp$.

\np By induction can find a sequence $\{x_n\}_{n \geq 1}$ and open balls $(B_n)$ s.t.
\begin{equation*}
    x_{n+1} \in B_{n+1} \subset \bar{B}_{n+1} \subset B_n,\ B_n \text{ has radius smaller than }\frac{1}{2^{n-1}},
\end{equation*}
and $B_n \cap A_n = \emp$.

\np The sequence $\{x_n\}$ is Cauchy, and $M$ is complete $\Rightarrow$ $x_\infty \in M$. Note that $x_\infty \in B_n,\ \forall\ n \geq 1$ $\Rightarrow$ $x_\infty \notin A_n,\ \forall\ n$.

\np Hence, $M \neq \cupp{n=1}{\infty} A_n$, and we reach a contradiction.
\end{proof}


\vspace{3pt}
\begin{theorem}[Open Mapping Theorem]\ \\
Let $X,Y$ be Banach spaces and $T \in \Ls(X,Y)$. Assume $T$ is surjective (i.e. $T(X) = Y$), then $T$ maps open sets in $X$ to open sets $Y$.
\end{theorem}
\begin{proof}\ \\
Let $B_x = \{x:\ \norm{x} \leq 1\}$ be a unit ball in $X$. Similarly $B_y$ be a unit ball in $Y$. It suffices to show 
\begin{equation*}
    T(B_x) \supset \vep B_Y,\ \text{for some } \vep > 0.
\end{equation*}
To see this let $U \subset X$, $U$ open and $y \in TU$. Need to show $TU$ contains a neighborhood of $y$. Let $x \in U,\ T x = y$, $U$ open $\Rightarrow$ $\exists\ \delta > 0$ s.t.
\begin{equation*}
    U \supset x + \delta B_X,\ TU \supset T(x + \delta B_X) = y+ \delta B_x \supset y + \delta \vep B_Y,
\end{equation*}
which show $TU$ contains neighborhood of $y$.

\np It remains to show $TB_X \supset \vep B_Y$ for some $\vep > 0$. Observe that 
\begin{equation*}
    X = \cupp{n=1}{\infty} n B_X\ \Rightarrow\ Y = TX = \cupp{n=1}{\infty} nTB_X.
\end{equation*}
Baire Category Theorem implies $\exists\ n \geq 1$ s.t. $n\bar{TB_X}$ has mom-empty interior for some $n$, and thus $\bar{TB_X}$ has non-empty interior. Hence $\exists\ y \in Y,\ \delta > 0$ s.t.
\begin{equation*}
    y + \delta B_Y \subset \bar{TB_X}.
\end{equation*}
$TX = Y\ \Rightarrow\ \exists\ x \in X$ s.t. $Tx = y$, and thus
\begin{equation*}
    \delta B_Y \subset \bar{T(B_X - x)}.
\end{equation*}
In addition, $B_X - x \subset n B_X$ for some $n \geq 1$, and thus
\begin{equation*}
    \delta B_Y \subset n \bar{TB_X},
\end{equation*}
which implies $\bar{TB_X \supset \vep B_Y}$ for some $\vep > 0$.

\np Finally show $\bar{TB_X} \subset T(2B_X)$ $\Rightarrow$ $TB_X \supset \frac{\vep}{2} B_Y$. For some $\vep > 0$ we have $\bar{TB_X} \supset \vep B_Y$, can conclude $\bar{TB_X} \subset T (2B_X)$.

\np Use scaling argument, let $y \in \bar{TB_X}$, then $\exists\ x_1 \in B_X$ s.t.
\begin{equation*}
    y - Tx_1 \in  \frac{\vep}{2} B_Y \subset T(\frac{1}{2} B_X).
\end{equation*}
Next choose $x_2 \in \frac{1}{2} B_X$ s.t.
\begin{equation*}
    y - Tx_1 - Tx_2 \in \frac{\vep}{4} B_Y \subset \bar{T(\frac{1}{4}B_X)}.
\end{equation*}
By induction construct sequence $\{x_n\}_{n \geq 1}$ s.t.
\begin{equation*}
    x_n \in \frac{1}{2^{n-1}}B_X,\ y - \summ{j=1}{n} Tx_j \in \frac{\vep}{2^n} B_Y.
\end{equation*}
Then $x = \summ{j=1}{\infty} x_n \in 2B_X$, and $Tx = y$.
\end{proof}


\vspace{3pt}
\begin{corollary}[Inverse Mapping Theorem]\ \\
Let $T:X \to Y$ be a bounded linear operator between Banach spaces $X$ and $Y$, which is both injective and surjective, then $T$ has a bounded inverse $T^{-1} \in \Ls(Y,X)$.
\end{corollary}


\vspace{3pt}
\begin{corollary}\ \\
Suppose $X,Y$ are Banach spaces, $T \in \Ls(X,Y)$, then the following are equivalent:
\begin{enumerate}[label = (\alph*)]
    \item $T$ is injective and $\Im{T}$ is closed.
    \item $T$ is bounded below: $\exists\ c > 0,\ \norm{Tx} \geq c \norm{x},\ x\in X$.
\end{enumerate}
\end{corollary}
\begin{proof}\ \\
\begin{itemize}
    \item $\Rightarrow:$\ \\
    $T^{-1}: \Im{T} \to X$ is bounded since $\Im{T}$ is Banach space. From Open Mapping Theorem, 
\begin{equation*}
    \norm{T^{-1} y} \leq c^{-1} \norm{y},\ y \in \Im{T},\ c > 0, \text{ some constant}.
\end{equation*}
Set $y = Tx$, we have $\norm{Tx} \geq c\norm{x},\ x \in X$. Hence (b) holds.
    \item $\Leftarrow:$\ \\
    Suppose (b) holds, we have $T$ is injective (i.e. $Tx = 0$ $\Rightarrow$ $x = 0$). To see this $\Im{T}$ closed, and let $(x_n) \subset X$ be a sequence s.t. $(Tx_n)$ is Cauchy. By
    \begin{equation*}
        \norm{Tx_n - Tx_m} \geq c\norm{x_n - x_m},\ \forall\ n,m,
    \end{equation*}
    we have $(x_n)$ is Cauchy, and hence $x_n \to x_\infty \in X$, which implies $Tx_n \to Tx_\infty \in \Im{T}$. Can conclude $\Im{T}$ is closed.
\end{itemize}
\end{proof}


\subsection{Closed Graph Theorem}

\begin{definition}\ \\
For $T \in \Ls(X,Y)$, $X,Y$ are Banach spaces. The graph of $T$ is defined as
\begin{equation*}
    \Gamma(T):= \{(x,Tx):\ x \in X\} \subset X \times Y.
\end{equation*}
It is closed provided it is a closed subspace of $X \times Y$. Hence if $\{x_n\}_{n \geq 1}$ is a sequence in $X$ s.t. both $\{x_n\}_{n \geq 1}$ and $\{Tx_n\}_{n \geq 1}$ are Cauchy, then $\exists\ x_\infty \in X$ s.t.
\begin{equation*}
    x_n \to x_\infty,\ Tx_n \to Tx_\infty\ \&\ y_\infty = Tx_\infty.
\end{equation*}
\end{definition}


\vspace{3pt}
\begin{proposition}[Closed Graph Theorem]\ \\
Let $T: X \to Y$ be a linear operator between Banach spaces $X$ and $Y$, then $T$ is bounded (continuous) if and only if $\Gamma(T)$ is closed.
\end{proposition}
\begin{proof}\ \\
$T$ bounded implies $\Gamma(T)$ closed trivially.

\np Now assume $\Gamma(T)$ is closed, which implies $\Gamma(T)$ is a Banach space. Then use Open Mapping Theorem. Define norm on $X \times F$ by $\norm{(x,y)} = \norm{x} + \norm{y}$, and $\Gamma(T)$ is Banach space with this norm. Define $u: \Gamma(T) \to X,\ u(x,Tx) = x,\ x \in X$, then $u$ is bounded $\norm{u} \leq 1$, and bijective.

\np Open Mapping Theorem implies $u^{-1}: X \to \Gamma(T)$ is bounded. Hence $\norm{u(x, Tx)} \geq c\norm{(x, Tx)},\ x \in X$ for some $c > 0$, i.e. 
\begin{equation*}
    \norm{x} \geq c[\norm{x} + \norm{Tx}]\ \Rightarrow\ \norm{Tx} \leq (\frac{1}{c} - 1) \norm{x},\ x \in X,
\end{equation*}
which implies $T$ is bounded.
\end{proof}


\vspace{3pt}
\begin{proposition}[Hellinger-Toeplitz]\ \\
$T: \Hs \to \Hs$ is a linear operator, which is \underline{self-adjoint}, i.e.
\begin{equation*}
    \inprod{Tx}{y} = \inprod{x}{Ty},\ x,y \in \Hs,
\end{equation*}
then $T$ is bounded.
\end{proposition}
\begin{proof}
Show that self-adjoint $\Rightarrow$ $\Gamma(T)$ is closed. Let $\{x_n\}_{n \geq 1} \in \Hs$ s.t.
\begin{equation*}
    x_n \to x_\infty \in \Hs,\ Tx_n \to y_\infty \in \Hs.
\end{equation*}
Need to show $Tx_\infty = y_\infty$. Use \underline{self-adjointness} of $T$. For $x \in \Hs$,
\begin{align*}
    \inprod{z}{y_\infty} &= \lim_{n \to \infty} \inprod{z}{Tx_n}\\
    &= \lim_{ n \to \infty} \inprod{Tz}{x_n}\\
    &= \inprod{Tz}{ x_\infty}\\
    &= \inprod{z}{Tx_\infty}
\end{align*}
holds for all $z \in \Hs\ \Rightarrow\ Tx_\infty = y_\infty$. Hence $\Gamma(T)$ is closed and $T$ is bounded.
\end{proof}


\vspace{3pt}
\begin{proposition}[Uniform Boundedness Theorem]\ \\
Let $X,Y$ be Banach spaces, and $\Ts \in \Ls(X,Y)$ a family of operators $X \to Y$ s.t.
\begin{equation*}
    \sup_{T \in \Ts} \norm{Tx} < \infty,\ \forall\ x \in X \ \Rightarrow\ \sup_{T \in \Ts} \norm{T} < \infty.
\end{equation*}
\end{proposition}
\begin{proof}\ \\
Define $M: X \to \R$ by
\begin{equation*}
    M(x) = \sup_{T \in \Ts} \norm{Tx},\ x \in X.
\end{equation*}
Then $X = \cupp{n=1}{\infty} X_n$ where $X_n = \{x \in X:\ M(x) \leq n\}$. By \hyperref[BCT]{Baire Category Theorem}, there exists $n \geq 1$ s.t.
\begin{center}
    $\bar{X_n}$ has non-empty interior.
\end{center}
Note the functional $x \to M(x),\ x \in X$ is lower semi-continuous, i.e. $M(x) \leq \liminf_{x_n \to x} M(x_n)$. Therefore,
\begin{equation*}
    \norm{Tx} \leq \lim_{n \to \infty} \norm{Tx_n} \leq \liminf_{n\to \infty}M(x_n).
\end{equation*}
Take sup over $x$, we get
\begin{equation*}
    M(x) \leq \liminf_{n \to \infty} M(x_n).
\end{equation*}

\np Hence, $X_n = \{x \in X:\ M(x) \leq n\}$ is closed. Thus $\bar{X_n} = X_n$, can conclude $X_n$ has non-empty interior. Therefore, $X_n \supset X_0 + \vep B_x$ for some $\vep > 0$, where $B_x = \{x \in X:\ \norm{x} \leq 1\}$. 

\np Note $M(\cdot)$ is symmetric and convex, we have
\begin{align*}
    M(-x) &= M(x),\ x \in X,\\
    M(\lambda x + (1-\lambda)y) &\leq \lambda M(x) + (1 - \lambda) M(y),\ x,y \in X,\ 0 < \lambda < 1.
\end{align*}
Hence $X_n \supset X_0 + \vep B_x$, and the symmetry implies $X_n \supset - X_0 + \vep B_x$. By convexity, we know $X_n \supset \vep B_x$. Therefore, we can conclude 
\begin{equation*}
    \norm{x} \leq \vep \Rightarrow \sup_{T \in \Ts} \norm{Tx} \leq n \Rightarrow \sup_{T \in \Ts} \norm{T} \leq \frac{n}{\vep}.
\end{equation*}
\end{proof}


\vspace{3pt}
\begin{corollary}[Weak Boundedness implies Strong Boundedness]\ \\
    Let $A \subset X$ and suppose $A$ is weakly bounded (i.e $\sup_{f \in X^*} \abs{f(x)} < \infty,\ \forall\ x \in A$), then $A$ is strongly bounded (i.e $\sup_{x \in A} \norm{x} < \infty$).
\end{corollary}
\begin{proof}\ \\
Embed $A$ into $A^** \subset X^{**}$, $x \to x^{**}$, 
\begin{equation*}
    \sup_{x^{**} \in A^{**}} \abs{x^{**} (f)} < \infty,\ \forall\ f \in X^*.
\end{equation*}
Hence uniform boundedness principle implies
\begin{equation*}
    \sup_{x^{**} \in A^{**}} \norm{x^{**}} < \infty.
\end{equation*}
\hyperref[HB thm]{Hahn Banach Theorem} implies
\begin{equation*}
    \norm{x^{**}} = \norm{x},\ x \in X.
\end{equation*}
\end{proof}


\subsubsection{Schauder Basis}

\begin{definition}
    Let $X$ be a separable Banach space, a sequence $\{x_k\}_{k \geq 1}$ is a \underline{Schauder basis} for $X$ if every $x \in X$ can be \textit{uniquely} represented as a convergent series $x = \summ{k=1}{\infty} a_k x_k,\ a_k \in \R \text{ or } \C$.  
\end{definition}

\vspace{3pt}
\begin{theorem}\ \\
Let $\{x_k\}_{k \geq 1}$ be a Schauder basis for Banach space $X$, then $\exists\ M \geq 0$ (called the Basis constant) with the property
\begin{equation*}
    \norm{\summ{k=1}{n} a_k x_k} \leq M \norm{x},\ x \in X,\ n = 1,2,\dots,
\end{equation*}
where $x = \summ{k=1}{\infty} a_k x_k$.
\end{theorem}
\begin{proof}\ \\
Define a sequence space
\begin{equation*}
    E = \{a = \{a_k\}_{k \geq 1}:\ \summ{k=1}{\infty} a_k x_k \text{ converges in } X\}.
\end{equation*}
For $a \in E$ define 
\begin{equation*}
    \norm{a} = \sup_{n \geq 1} \norm{\summ{k=1}{n} a_k x_k} < \infty,
\end{equation*}
$\norm{\cdot}$ is a norm on $E$.

\np Note that $\norm{a} = 0 \Rightarrow a = 0$, which follows from uniqueness property of Schauder basis, and thus $E$ is a Banach space, i.e. $E$ is complete.

\np Define operator $T: E \to X$ by $Ta = \summ{k=1}{\infty} a_k x_k$, we have $\norm{Ta} \leq \norm{a}$ so $T$ is bounded and also bijective. Open Mapping Theorem implies $T^{-1}: X \to E$ is bounded and thus
\begin{equation*}
    \norm{T^{-1}} \leq M < \infty\ \Leftrightarrow\ \norm{Ta} \geq \frac{1}{M} \norm{a},\ a \in E,
\end{equation*}
i.e. $\sup_{n \geq 1} \norm{\summ{k=1}{n} a_k x_k} \leq M \norm{\summ{k=1}{\infty} a_k x_k}$.
\end{proof}


\np For the partial sum operator $S_n: X \to X,\ x \to \summ{k=1}{n} a_k x_k$, where $x = \summ{k=1}{\infty} a_k x_k$, we have shown that $S_n$ is a bounded operator and $\sup_{n \geq 1} \norm{S_n} < \infty$. Observe that $a_k = a_k(x)$ is a linear functional on $X$. They are called \underline{Biorthogonal Fucntionals} of the basis $\{x_k\}_{k \geq 1}$ and denoted by $x_k^*,\ k \geq 1$. Show $x_k^* \in X^*$, i.e. a bounded linear functional.  To do this write $x_n^*(x) x_n = S_n(x) - S_{n-1}(x),\ n = 1,2,\dots$. Previous theorem implies
\begin{equation*}
    \norm{x_n^*(x) x_n} \leq \norm{S_n(x)} + \norm{S_{n-1}(x)} \leq 2M \norm{x}.
\end{equation*}
We conclude $x_n^* \in X^*$ and $\sup_{n \geq 1} \norm{x_n^*} \norm{x_n} < \infty$. 

\vspace{12pt}
\subsubsection{Compactness}

\begin{definition}[Compactness]\ \\
A subset $A$ of a topological space is compact if every open cover of $A$ has a finite subcover, i.e. if $A \subset \cup_{\alpha} U_\alpha$ for some collection of open sets $U_\alpha$, then $A \subset \cup_{k=1}^n U_{\alpha_k}$ for some finite subcollection.
\end{definition}

\vspace{3pt}
\begin{proposition}[Properties of compact sets]\ 
\begin{enumerate}[label = (\alph*)]
    \item Compact sets of a Hausdorff space are closed.
    \item Closed subsets of compact sets are compact.
    \item The image of a compact set under a continuous function is compact.
    \item Continuous functions on compact sets are uniformly continuous, and attain the their maximum and minimum.
\end{enumerate}
\end{proposition}


\vspace{3pt}
\begin{definition}[Precompact set]\ \\
A set $A$ is \textit{precompact} if its closure $\bar{A}$ is compact.
\end{definition}

\vspace{3pt}
\begin{definition}[$\vep$-nets]\ \\
Let $A$ be a subset of a metric space $X$, a subset $\Omega_{\vep} \subset X$ is an \underline{$\vep$-net} for $A$ if $A$ can be covered by balls of radius $\vep$ centered at points on $\Omega_\vep$ if
\begin{equation*}
    A \subset \{y: d(y,x) < \vep \text{ for some } x \in \Omega_\vep\}.
\end{equation*}
\end{definition}


\vspace{3pt}
\begin{theorem}\ \\
Let $A$ be a subset of a complete metric space $X$, then the following are equivalent
\begin{enumerate}[label = (\alph*)]
    \item $A$ is precompact.
    \item Every sequence $\{x_n\}$ in $A$ has a Cauchy subsequence (which converges in $X$).
    \item For every $\vep > 0$, $\exists$ an finite $\vep$-net for $A$.
\end{enumerate}
\end{theorem}


\vspace{3pt}
\begin{theorem}[Heine-Borel]\ \\
A subset $A$ of a finite dimensional normed space $X$ is precompact if and only if $A$ is bounded.
\end{theorem}


\vspace{3pt}
\begin{lemma}[Approximation by finite dimensional subspaces]
A subset $A$ of a normed space $X$ is precompact if and only if $A$ is bounded and for every $\vep > 0$, there exists a finite dimensional subspace $Y_\vep$ of $X$ containing the $\vep$-net for $A$.
\end{lemma}
\begin{proof}\ 
\begin{itemize}
    \item \textbf{Necessity:}
    Let $A$ be precompact and $\vep > 0$, then $\exists$ a finite $\vep$-nect $\Omega_\vep$ for $A$. Now, take $Y_\vep = \vspan{\Omega_\vep}$.
    \item \textbf{Sufficiency:} Since $A$ is bounded $A \subset R B_X$ for some $R > 0$, where $B_X = \text{ unit ball } \{x \in X:\norm{X} \leq 1\}$. Hence we can restrict to points of $\Omega_\vep$ of $\Omega_\vep$ contained in $(R + \vep) B_{Y_\vep}$. Therefore,
    \begin{equation*}
        A \subset \{x \in X: d(x, (R + \vep) B_{Y_\vep}) < \vep\}.
    \end{equation*}
    $(R + \vep) B_{Y_\vep}$ is compact, and $(R+\vep)B_{Y_\vep}$ is covered by finite collection of balls of radius $\vep$.
\end{itemize}
\end{proof}


\vspace{3pt}
\begin{theorem}[F. Riesz]\ \\
The unit ball $B_X$ of an $\infty$-dimensional normed space if never compact.
\end{theorem}
\begin{proof}\ \\
Suppose $B_X = \{x \in X: \norm{x} \leq 1\}$ is compact, then by lemma, we can find a finite-dimensional subspace $Y$ containing an $\vep = \frac{1}{2}$-net for $B_X$, i.e. $d(x,Y) \leq \frac{1}{2},\ \forall\ x \in B_X$. Note $X$ is $\infty$-dim and $Y$ is finite-dim, which implies the quotient space $\frac{X}{Y}$ is nontrivial. Note $Y$ is a closed subspace of $X$. Hence norm on $X$ induces a norm on $\frac{X}{Y}$. Let $x \in X,\ [x] \in \frac{X}{Y}$ s.t. $\norm{[x]} = 0.9$, where $\norm{[x]} = \int_{y \in Y} \norm{x - y}$. Choose $\bar{y} \in Y$ s.t. $\norm{x - \bar{y}} \leq 1$, which implies $x - \bar{y} \in B_X$ and $d(x - \bar{y},Y) = 0.9 > \frac12$.
\end{proof}





