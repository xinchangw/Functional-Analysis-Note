\section{Banach Spaces}
\subsection{Banach Space}
\begin{definition}[Banach Space]\ \\
A linear normed space is a \underline{Banach space} if it is complete (i.e. every Cauchy sequence converges).
\end{definition}
\begin{remark}\ \\
Completeness means if $x_n \in E,\ n \geq 1$ is a Cauchy sequence has property:
$$\lim_{m \to \infty} \sup_{n \geq m} \norm{x_n - x_m} = 0$$
then $\exists\ x_\infty \in E$ s.t. $\lim_{n \to \infty} \norm{x_n - x_\infty} = 0$.
\end{remark}
\begin{example}
The spaces $\ell_1, \ell_\infty$ and $C(K)$ are \underline{Banach spaces}.
\end{example}

\begin{remark}\ \\
Let $E$ be a linear normed space and $\{x_k\}_{k=1}^\infty$ a sequence in $E$. We say the sequence is \underline{absolutely summable} if $\sum_{k  =1}^\infty \abs{x_k} < \infty$.
\end{remark}


\vspace{3pt}
\begin{theorem}[Criterion for completeness]\ \\
A normed space $(E, \norm{\cdot})$ is a Banach space \underline{if.f} every \underline{absolutely summable} sequence in $E$ converges.
\end{theorem}
\begin{proof}\
\begin{itemize}
    \item $\Rightarrow:$\ \\ 
    Suppose $E$ is a \underline{Banach space} and $\{x_k\}_{k=1}^\infty$ a \underline{absolutely summable} sequence. Set $S_n = \sum_{k = 1}^n x_k,\ n \geq 1$. Show $S_n$ is Cauchy, and thus $E$ complete implies $\exists\ S_\infty$ s.t. $\lim_{n \to \infty} \norm{S_n - S_\infty} = 0$.\\ For $n\geq m$, 
    \begin{equation*}
        \norm{S_n - S_m} = \norm{\sum_{k = m+1}^n x_k} \leq \sum_{k = m+1}^n \norm{x_k} \leq \sum_{k = m+1}^\infty \norm{x_k}.
    \end{equation*} 
    Therefore, $\lim_{m \to \infty} \sum_{k = m+1}^\infty \norm{x_k} = 0$, which shows the sequence $\{x_k\}_{k = 1}^\infty$ is Cauchy.
    \item $\Leftarrow:$\ \\ 
    Suppose $E$ is not complete, then $\exists$ a Cauchy sequence $\{x_n\}_{n=1}^\infty$ which does not converge. Furthermore, no subsequence of $\{x_n\}_{n=1}^\infty$ converges. Construct an \underline{absolutely summable} sequence which does not converge.\\
    Define $n(1) \geq 1$ s.t. $\norm{x_n - x_{n(1)}} \leq \frac{1}{2}$ $\forall\ n > n(1)$, and let $n(2) > n(1)$ be s.t. $\norm{x_n - x_{n(2)}} \leq \frac{1}{2^2}$ $\forall\ n > n(2)$. Construct let this and get $n(1), n(2), n(3),\dots$ s.t. $\norm{x_n - x_{n(k)}} \leq \frac{1}{2^k},\ \forall\ n > n(k)$. Define $w_j = x_{n(j+1)} - x_{n(j)},\ j = 1,2,\dots$, then
    $$x_{n(m)} = x_{n(1)} + \sum_{j = 1}^{m-1} w_j,\ m = 1,2,\dots$$
    $\{x_{n(m)}\}$ does not converge. Hence, series $\sum_{j = 1}^{m-1} w_j$ does not converge. However,
    $$\sum_{j = 1}^\infty \norm{w_j} \leq \sum_{j = 1}^\infty \frac{1}{2^j} = 1$$
    which shows $\{w_j\}$ is \underline{absolutely summable}, and thus $\sum_{j = 1}^m w_j$ converges, which reachs a contradiction.
\end{itemize} 
\end{proof}

\clearpage
\subsection{Completion of normed spaces to a Banach space}
\begin{theorem}\ \\
Suppose $E$ is a normed space, then $\exists$ a \underline{Banach space} $\hat{E}$ called a completion of $E$ with the following properties:\\
$\exists$ a linear map $i: E \to \hat{E}$ s.t. 
\begin{enumerate}[label = (\alph*)]
    \item $\norm{ix} = \norm{x},\ x \in E$.
    \item $\Im(i)$ is dense in $\hat{E}$.
\end{enumerate}
\end{theorem}
\begin{proof}
The core of the proof has been presented in HW1 Pr4, and the remaining thing is to show $\hat{E}$ is a \underline{Banach space}. \textcolor{red}{Put them here later}.

\np Denote $(x_n)$ the Cauchy sequence in $E$, and $[(x_n)^N]$ the Cauchy sequence in $\hat{E}$. For each $N$, $\exists\ n_N > 0$ s.t. $\norm{x_n^N - x_m^N}_E \leq \frac{1}{N},\ \forall\ m,n \geq n_N$. Choose $y_N := x_{n_N}^N$, we construct the sequence $[y_N] = [(x_{n_N})^N]$. Note that $\forall\ \ep > 0$, $\exists\ N_\ep$ s.t.
\begin{equation*}
    \norm{(x_n)^N - (x_n)^M}_{\hat{E}} = \lim_{n \to \infty} \norm{x_n^N - x_n^M}_E  \leq \ep,\ \forall\ N,M \geq N_\ep \geq \frac{1}{\ep}.
\end{equation*}
Thus we can further find $n_\ep > 0$ such that
\begin{equation*}
    \norm{x_n^N - x_n^M}_E \leq 2 \ep,\ \forall\ n \geq n_\ep.
\end{equation*}
Taking $n = \{n_\ep,n_N,n_M\}$, we have
\begin{equation*}
    \norm{y_N - y_M}_E \leq \norm{y_N - x_n^N}_E + \norm{x_n^N - x_n^M}_E + \norm{x_n^M - y_M}_E \leq\frac{1}{N} + 2\ep  + \frac{1}{M} \leq  4\ep.
\end{equation*}
therefore, $\forall\ \ep > 0$, $\exists\ N_\ep$ s.t. $\norm{y_N - y_M}_E \leq 4\ep,\ \forall\ N,M \geq N_0$, which shows $[y_N]$ is a Cauchy sequence in $E$. Finally we want to show that $[(x_n)^N]$ converges to $[y_N]$. Given $\ep > 0$, can choose $n_\ep > 0$ s.t.
\begin{equation*}
    \norm{y_n - y_m} < \ep,\ \forall\ n,m \geq n_\ep,
\end{equation*}
then we can pick $N_0 \geq \max{\{1/\ep, n_\ep\}}$. For all $N \geq N_0$,
\begin{equation*}
    \norm{x_n^N - y_N}_E = \norm{x_n^N - x_{n_N}^N}_E \leq \frac{1}{N} \leq \frac{1}{N_0} \leq \ep,\ \forall\ n \geq n_N. 
\end{equation*}
Fix $N \geq N_0$, then
\begin{equation*}
    \norm{x_n^N - y_n}_E \leq \norm{x_n^N - y_N}_E + \norm{y_N - y_n}_E \leq 2\ep,\ \forall\ n \geq \max{\{n_N,n_\ep\}}.
\end{equation*}
Therefore,
\begin{equation*}
    \norm{[(x_n)^N] - [y_n]}_{\hat{E}} = \lim_{n \to \infty} \norm{x_n^N - y_n}_E \leq 2\ep
\end{equation*}
Since $\ep > 0$ is arbitrary, we complete the proof.
\end{proof}


\begin{remark}\ \\
    $i$ is an \underline{isometric} embedding of $E$ into $\hat{E}$ (Property (a)). $\hat{E}$ is the smallest \underline{Banach space} containing image of $i$ over $E$ (Property (b)). 
\end{remark}


\vspace{3pt}
\begin{theorem}[$\ell_p$ space is complete]\ \\
For $1 \leq p \leq \infty$, $\ell_p$ space is a \underline{Banach space}.
\end{theorem}
\begin{proof}\ \\
Suppose $\{x^n\}_{n \in \N}$ is a Cauchy sequence in $\ell_p$, where $x^n = \{x^n_k\}_{k \in \N}$. Then $\forall\ \vep > 0$, $\exists\ N \in \N$ s.t. (for $1 \leq p < \infty$, and $p = \infty$ case is trivial)
\begin{equation*}
    \norm{x^n - x^m}_p = (\summ{k=1}{\infty} \abs{x^n_k - x^m_k}^{p})^{1/p} < \vep,\ \forall\ n > m \geq N.
\end{equation*}
The above equation implies that for fixed $k$, $\{x_k^n\}_{n \in \N}$ forms a Cauchy sequence in $\R$, and thus $x_k^n \to x_k^\infty$ for some $x^\infty_k \in \R$, which implies $x^n \to x^\infty$. It remains to show
\begin{itemize}
    \item $x^\infty \in \ell_p:$\\
    Every Cauchy sequence is bounded, and thus $\exists\ M > 0$ s.t.
    \begin{equation*}
        \norm{x^n}_p = (\summ{k=1}{\infty} \abs{x^n_k}^p)^{1/p} \leq M,\ \forall\ n \in \N.
    \end{equation*}
    Therefore, we have
    \begin{equation*}
        \summ{k=1}{m} \abs{x^n_k}^p \leq M^p,\ \forall\ m,n \in \N.
    \end{equation*}
    Take $n \to \infty$ and then $m \to \infty$, we complete the claim.
    \item $x^n \to x^\infty$ in $\ell_p$:\\
    Fix $n > N$, and note that $\summ{k=1}{\infty} \abs{x^n_k - x^m_k} < \vep^p,\ \forall\ m > n$, which implies for each $K$,
    \begin{equation*}
        \summ{k=1}{K} \abs{x^n_k - x^m_k} < \vep^p.
    \end{equation*}
Take $m \to \infty$, we have
    \begin{equation*}
        \summ{k=1}{K} \abs{x^n_k - x^\infty_k} < \vep^p.
    \end{equation*}
    Finally take $K \to \infty$, we complete the proof.
\end{itemize}
\end{proof}
\begin{remark}\ \\
    For $1 < p < \infty$, $x = \{x_n\}_{n = 1}^\infty \in \ell_p$ if $\sum_{n=1}^\infty \abs{x_n}^p < \infty$, $\norm{x}_p = (\sum_{n=1}^\infty \abs{x_n}^p)^{1/p}$.\\
    Show that $x \to \norm{x}_p$ satisfies properties of a norm:
    \begin{itemize}
        \item $\norm{x} \geq 0$, $\norm{x} = 0$ $\Leftrightarrow$ $x = 0$ (easy).
        \item $\norm{\lambda x} = \abs{\lambda} \norm{x}, \lambda \in \F$ (easy).
        \item Triangle inequaility: Recall \hyperref[1.10]{Cor 1.10}, it suffices to show $B(0,1)$ is convex. Note $f(x) = \sum_{n=1}^\infty \abs{x_n}^p$ is a convex function, since $x \to \abs{x}^p,\ x \in \R$ is convex for $p \geq 1$.
    \end{itemize}
    Hence $\norm{x + y}_p \leq \norm{x}_p + \norm{y}_p$ and
        \begin{equation*}
            (\sum_{j=1}^\infty \abs{x_j + y_j}^p)^{1/p} \leq (\sum_{j=1}^\infty \abs{x_j}^p)^{1/p} + (\sum_{j=1}^\infty \abs{y_j}^p)^{1/p} \quad (\text{Minkowski Inequality})
        \end{equation*}
    The usual proof of \underline{Minkowski Inequality} is via the \underline{H\"older Inequality}.
    \begin{lemma}[H\"older Inequality for $\ell_p$ space]
    \begin{equation*}
        \sum_{j = 1}^\infty \abs{x_j y_j} \leq (\sum_{j = 1}^\infty \abs{x_j}^p)^{1/p} (\sum_{j = 1}^\infty \abs{y_j}^q)^{1/q}
    \end{equation*}
, where $\frac{1}{p} + \frac{1}{q} = 1.$
    \end{lemma}
\begin{proof}
\begin{itemize}
    \item Consider $F(t) = t - \frac{t^p}{p},\ t \geq 0$, we have by finding the maximum
    \begin{equation*}
        t \leq \frac{t^p}{p} + 1 - \frac{1}{p} = \frac{t^p}{p} + \frac{1}{q},\ \frac{1}{p} + \frac{1}{q} = 1.
    \end{equation*}
    \item Taking $t = \frac{\alpha}{\beta^{q - 1}}$, we have
    \begin{equation*}
        \alpha \beta \leq \frac{\alpha^p}{p} + \frac{\beta^q}{q}.
    \end{equation*}
    which is known as \underline{Yound's Inequality}.
    \item WLOG, assume $0 < \norm{f}_p, \norm{g}_q < \infty$. Consider $F(x) = \frac{f(x)}{\norm{f}_p},\ G(x) = \frac{g(x)}{\norm{g}_p}$. Note $\norm{F}_p = \norm{G}_p = 1$, we have
    \begin{align*}
        \int \abs{F(x) G(x)} d\mu &\leq \int \frac{\abs{F(x)}^p}{p} + \frac{\abs{G(x)}^q}{q} d\mu\\
        \frac{\norm{fg}_1}{\norm{f}_p \norm{g}_q} &\leq \frac{1}{p} + \frac{1}{q} = 1
    \end{align*}
    which is a more general case of \underline{H\"older Inequality}. 
\end{itemize}
\end{proof}
\begin{remark}\ \\
    The \underline{Young's Inequality} is a special case of the inequality $$xy \leq f(x) + \Ls f(y)$$
where $\Ls f(y) := \sup_x [xy - f(x)]$ is the \underline{Legendre Transform} of $f(\cdot)$.\\
If $f(\cdot)$ is convex, then $F_N: (x,y) \to xy - f(x)$ is concave and has unique maximum.
\end{remark}
\begin{remark}\ \\
    $\Ls f(\cdot)$ is always convex even if $f(\cdot)$ is not. $f(x) = \frac{x^p}{p}$ then $\Ls f(y) = \frac{y^{p'}}{p'}$
\end{remark}
\begin{proof}[Proof of \underline{Minkowski Inequality}]\
\begin{equation*}
    \sum_{j = 1}^\infty \abs{x_j + y_j}^p \leq \sum_{j=1}^\infty \abs{x_j} \abs{x_j + y_j}^{p-1} + \sum_{j= 1}^\infty \abs{y_j} \abs{x_j + y_j}^{p-1}
\end{equation*}
and by H\"older inequality, we have 
\begin{equation*}
    \sum_{j = 1}^\infty \abs{x_j} \abs{x_j + y_j}^{p-1} \leq (\sum_{j= 1}^\infty  \abs{x_j}^p)^{1/p} (\sum_{j = 1}^\infty \abs{x_j + y_j}^{(p - 1) p'})^{1 /p'},
\end{equation*}
where $\frac{1}{p} + \frac{1}{p'} = 1$. Then proceed to complete the proof.
\end{proof}
\end{remark}

\np Let $(\Omega, \Sigma,\mu)$ be a measure space and $\Ls_p(\Omega, \Sigma, \mu)$ be all measurable functions $f: \Omega \to \F$ s.t. $\int_\Omega \abs{f}^p \dr\mu < \infty$. Then $\Ls_p (\Omega, \Sigma, \mu)$ is a normed space with norm $\norm{f}_p = (\int_\Omega \abs{f}^p \dr \mu)^{1/p}$.

\vspace{3pt}
\begin{theorem}\ \\
$\Ls_p(\Omega, \Sigma, \mu)$ is a \underline{Banach space}.
\end{theorem}
\begin{proof}\ \\
Let $\{f_n\}_{n = 1}^\infty$ be an \underline{absolutely summable} sequence in $\Ls^p$, then $\norm{\sum_{k = 1}^N f_k}_p \underbrace{\leq}_{\text{Minkowski}} \sum_{k = 1}^N \norm{f_k}_p \leq C $ (Independent of $N$). Hence $\int_\Omega \abs{\sum_{k=1}^N f_k}^p \dr\mu \leq C^p$.\\
\textbf{Case 1:} Assume all $f_k$ is non-negative. then the Monotone Convergence Theorem implies
$$\lim_{n \to \infty} \int_\Omega (\sum_{k=1}^N f_k)^p \dr\mu = \int_\Omega (\sum_{k=1}^\infty f_k)^p \dr\mu \leq C^p$$
Hence $g = \sum_{k=1}^\infty f_k \in \Ls_p$. Then we need to show $\sum_{k=1}^N f_k \to g$ in $\Ls_p$. Set $S_n = \sum_{k = n + 1}^\infty f_k$, then $S_n$ is a decreasing sequence $S_n \to 0$ a.e.. Conclude $\lim_{n \to \infty} \norm{S_n}_p = 0$ by DCT, as $\int_\Omega S_1^p \dr \mu < \infty$.\\
\textbf{Case 2:} For arbitrary $f_k: \Omega \to \R$, write $f_k = f_k^+ + f_k^-,\ f_k^+ = \sup{(f_k, 0)},\ f_k^- = \inf(f_k, 0)$. The sequence $\{f_k^+\}_{k = 1}^\infty$ is \underline{absolutely summable}. Proceed as before similarly.
\end{proof}