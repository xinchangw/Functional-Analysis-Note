\section{Bounded Linear Functionals}
\begin{definition}\ \\
Let $E = \underline{\text{Linear space over } \R \text{ or } \C}$. A linear functional on $E$ is a linear operator $f: E \to \R \text{ or } \C$ s.t. 
$$f(ax + by) = af(x) + bf(y),\ x,y \in E,\ a,b \in \R\text{ or }\C.$$ 
We say $f(\cdot)$ is a \underline{Bounded Linear Functional} if 
$$\norm{f} = \sup_{\norm{x} = 1} \abs{f(x)} < \infty.$$ 
By dilation and additive, properties of $f(\cdot)$ boundedness implies $\abs{f(x - y)} \leq \norm{f} \norm{x - y},\ x,y \in E$. Hence, $f(\cdot)$ is continuous and in fact Lipschitz continuous. Conversely, if a linear functional is continuous then it is bounded. 
\end{definition}
\begin{proof}\ \\
Suppose $f(\cdot)$ is not bounded $\Rightarrow$ $\exists$ sequence $(x_n) \subset E$ s.t. $\abs{f(x_n)} \geq n \norm{x_n},\ n=1,2,\dots$. By linearity, $\abs{f(\frac{x_n}{n \norm{x_n}})} \geq 1,\ n=1,2,\dots$, but $\lim_{n \to \infty} \frac{x_n}{n\norm{x_n}} = 0$ and $f(0) = 0$ $\Rightarrow$ $f(\cdot)$ is not continuous at $0$.
\end{proof}

\vspace{3pt}
\begin{definition}\ \\
    Let $E$ be a normed space, then space of all \underline{bounded linear functionals} on $E$ is known as the \underline{dual space $E^*$ of $E$}. It is also a normed space with norm $\norm{f} = \sup_{\norm{x} = 1} \abs{f(x)}$, and in fact a Banach space.
\end{definition}
\begin{remark}\ \\
A well-known proposition is that for any space $B(X,Y)$, where $Y$ is Banach, then $B(X,Y)$ is also Banach.
\end{remark}

\vspace{3pt}
\begin{definition}\ \\
    Let $E$ be a linear space and $H \subset E$ a subspace. We say $H$ is a hyperplane if $\text{codim}(H) = 1$, i.e. $\dim(\frac{E}{H}) = 1$. 
\end{definition}

\np \textbf{\underline{Goal:}} Make an equivalence between \underline{Bounded Linear Functionals} on $E$ and \underline{closed} hyperplanes in $E$. Does there exists a non closed hyperplane? (Not in finite dimensions) Another question is that does there exists a subset $F \subset \R$ which is not Lebesgue
measurable?\\ 
The answers are yes in both cases. However, construction uses axiom of choice.

\begin{proposition}\label{4.4}\ \\
$E$ a linear space,
\begin{enumerate}[label = (\alph*)]
    \item For every linear functional $f$ on $E$, $ker(f)$ is a hyperplane in $E$.
    \item  If $f,g\neq 0$ are linear functionals on $E$ s.t. $ker(f) = ker(g)$, then $f = ag$ for some $a \neq 0$. 
    \item For every hyperplane $H \subset E$, $\exists$ a linear functional $f \neq 0$ on $E$ s.t. $ker(f) = H$.
    \item If $E$ is a Banach space, then $f(\cdot)$ is Bounded \underline{if.f} $ker(f) = H$ is closed.
\end{enumerate}
\end{proposition}
\begin{proof}\ 
\begin{enumerate}[label = (\alph*)]
    \item Let $x,y \notin \ker(f)$ so $f(x), f(y) \neq 0$. Hence $\exists$ scalar $\lambda \neq 0$ s.t. $f(x) = \lambda f(y)$ $\Rightarrow$ $x - \lambda y \in \ker(f)$.\\
    Hence, $[x],[y] \in \frac{E}{\ker(f)}$, then $[x] = \lambda[y] \Rightarrow \dim[\frac{E}{\ker(f)}] = 1$\\
    $f$ is bounded $\Rightarrow$ f continuous $\Rightarrow$ $\ker(f) = f^{-1}(0)$ closed. 
    \item Consider the induced functional $\Tilde{f}, \Tilde{g}: \frac{E}{H} \to \R\text{ or } \C$, $H = \ker(f) = ker(g)$.\\ We have $\dim[\frac{E}{H}] = 1$ \imply $\Tilde{f} = a \Tilde{g}$ for some $a \neq 0$ \imply $f = ag$.
    \item Assume $\dim[\frac{E}{H}] = 1$ \imply $\frac{E}{H} = \{a [x_0]: a \in \C \text{ or } \R\}$ for some $x_0 \in E$.\\
    For $x \in E$, we have $[x] = a(x) [x_0]$ for some $a(x) \in \R \text{ or } \C$. Define $f(x) = a(x)$, then $f$ is linear and $\ker(f) = H$. Assume $E$ a Banach space and $H$ closed, we have $\dim(\frac{E}{H}) = 1$. Recall that $\frac{E}{H}$ is also a Banach space with norm
    $$\norm{[x]} = \inf_{y \in H} \norm{x + y}, x\in E $$
    Let $\Tilde{f}$ be a linear functional on $\frac{E}{H}$. Then we have $\dim(\frac{E}{H})$ finite \imply $\Tilde{f}$ continuous \imply $\abs{\Tilde{f}([x])} \leq A \norm{[x]},\ \forall\ x \in E$. Define $f(x) = \Tilde{f}([x]),\ x \in E$, then $\ker(f) = H$ and $\abs{f(x)} \leq A \norm{[x]} \leq A \norm{x}$.
\end{enumerate}
\begin{remark}
Recall "Every linear mapping on finite-dimensional vector space is continuous".
\end{remark}
\end{proof}



\vspace{12pt}
\subsection{Riesz Representation Theorem}

\begin{theorem}[Riesz Representation Theorem]\label{RRT}\ \\
Let $\Hs$ be a Hilbert space.
\begin{enumerate}[label = (\alph*)]
    \item For every $y \in \Hs$, the fucntion $f(x) = \inprod{x}{y},\ x \in \Hs$ is a bounded linear functional on $\Hs$.
    \item If $f: \Hs \to \C \text{ or } \R$ is a bounded linear functional on $\Hs$, then $\exists\ y \in \Hs$ s.t. $f(x) = \inprod{x}{y},\ x\in \Hs$. Hence the dual $\Hs^*$ of $\Hs$ is isometric to $\Hs$.
\end{enumerate}
\end{theorem}
\begin{proof}\
\begin{enumerate}[label = (\alph*)]
    \item It follows from Cauchy-Schwarz s.t.
    \begin{equation*}
        \abs{f(x)} = \abs{\inprod{x}{y}} \leq \norm{x} \norm{y} 
    \end{equation*}
    and we have $\norm{f} = \norm{y}$ by setting $x = \frac{y}{\norm{y}}$. 
    \begin{remark}\ \\
    In this case, the supremum of norm is attained by $\norm{f} = \sup_{\norm{x} = 1} \abs{f(x)} = f(\frac{y}{\norm{y}})$. It is also true for any \underline{finite} dimensional space, since unit ball is compact. In general it is not true for $\infty$-dimensional Banach spaces where the unit ball is not compact.
    \end{remark}
    \item Let $f: \Hs \to \R \text{ or } \C$ be a bounded linear functional on $\Hs$. Let $H = \ker(f)$, and $H$ is closed.\\ Let $H^\perp$ be orthogonal complement of $H$ s.t. $\Hs = H \bigoplus H^\perp$. We have $\dim[\frac{\Hs}{H}] = 1$ \imply $\dim(H^\perp) = 1$.\\
    Therefore, we can write
    \begin{equation*}
        \Hs = H + \vspan{y_0},\ \text{ for some } y_0 \in H^\perp.
    \end{equation*}
    Consider the map
    \begin{equation*}
        g(x) := \inprod{x}{y_0},
    \end{equation*}
    by part (a) we have
    \begin{equation*}
        g \in \Hs^*,\ \ker{g} = \ker{f}.
    \end{equation*}
    Therefore, by \hyperref[4.4]{Proposition 4.4}, we have $f = ag$ for some constant $a$, and it follows that
    \begin{equation*}
        f(x) = ag(x) = \inprod{x}{a y_0}.
    \end{equation*}
    Complete the proof.
\end{enumerate}
\end{proof}

\vspace{3pt}
\begin{theorem}[Radon-Nikodym Theorem]\label{R-N thm}\ \\
Let $\mu,\nu$ be two $\sigma$-finite measures s.t. $\nu \ll \mu$, i.e. $\nu$ is \underline{absolutely continuous} w.r.t. $\mu$. (i.e. $\mu(A) = 0$ \imply $\nu(A) = 0$) Then $\exists\ g \geq 0$ s.t. $g$ is $\mu$-integrable and $\nu(A) = \int_A g d\mu$ for measurable $A$, where $g$ is referred to as R-N Derivative $g = \frac{d\nu}{d \mu}$.
\end{theorem}
\begin{proof}\ \\
Consider the linear functional $F: \Ls^2(\mu) \to \R \text{ or } \C$ s.t. 
\begin{equation*}
    F(f) = \int_\Omega f d\mu.
\end{equation*}
Then we have $\norm{F(f)} \leq \norm{f}_2 \mu(\Omega)^{\frac{1}{2}}$ by H\"older inequality. $F$ is also a bounded linear functional on $\Ls^2(\mu + \nu)$, hence, the \hyperref[RRT]{Riesz Representation Theorem} implies $\exists\ h \in \Ls^2(\mu+\nu)$ s.t.
$$F(f) = \int f h d(\mu+\nu)$$ 
for $f \in \Ls^2(\mu + \nu)$, i.e. 
\begin{equation}
    \int_\Omega f d\mu = \int_\Omega fh d\mu + \int_\Omega fh d\nu.\label{eq:4.1}
\end{equation} 
Rearranging terms, we have
\begin{equation}
    \int_\Omega fh d\nu = \int_\Omega f(1 - h) d\mu.\label{eq:4.2}
\end{equation} 
Let $A \subset \Omega$, $A$ measurable, set $f = \frac{1}{h} \1_A$, we have $\nu(A) = \int_A  g d\mu,\ g = \frac{1-h}{h}$ \imply $g = \frac{d\nu}{ d \mu}$. It remains to show this is well-defined, i.e. $h \neq 0$.
\begin{claim}
$0 < h \leq 1$ a.e. \\
Note $\mu(A) = 0 \Leftrightarrow \mu(A) + \nu(A) = 0$. Let $A = \{h \leq 0\},\ f = \1_A$. \hyperref[eq:3.1]{Equation 3.1} implies that we have 
\begin{equation*}
    \int_A h [d\mu + d\nu] \leq 0 \Rightarrow \mu(A) = 0 \Rightarrow h > 0\ \mu-a.e.
\end{equation*}\\
To show $h \leq 1$, set $B = \{h > 1\}$, and let $f = \1_B$. \hyperref[eq:4.1]{Equation 4.1} implies 
\begin{equation*}
    \mu(B) = \int_B h [d\mu + d\nu] > \mu(B) \text{ unless } \mu(B) = 0
\end{equation*}
which shows that $h \leq 1$.
\end{claim}
Now that we know $0 < h \leq 1$, we have \hyperref[eq:4.2]{Equation 4.2} holds for all $f \geq 0,\ f \in \Ls^2(\mu + \nu)$. By using Monotone Convergence Theorem, we can conclude that \hyperref[eq:4.2]{Equation 4.2} holds for all $f \geq 0$ (Thus not restrict on finite measure $\mu,\nu$, but extend to $\sigma$-finite), and both sides could be $+\infty$. Now can plug in $f = \frac{1}{h}\1_A$ to get the result.\\ 
\end{proof}
\begin{remark}\ \\
R-N derivative $\frac{d\nu}{d\mu}$ is \underline{unique} since $\int_A g d\mu = 0$ for all $\mu$-measurable $A$ \imply $g = 0$ $\mu$-a.e
\end{remark}

\begin{remark}\ \\
For $\Hs = \text{Hilbert space}$, \hyperref[RRT]{Riesz Representation Theorem} identifies the dual space $\Hs^*$ and can imply R-N Theorem.
\end{remark}

\vspace{3pt}
\begin{corollary}\ \\   
Consider spaces $\Ls^p(\Omega, \mu)$ for
$1 \leq p < \infty$. R-N Theorem implies dual of $\Ls^p(\Omega, \mu)$ is (isometric to) $\Ls^{p'}(\Omega, \mu)$, where $\frac{1}{p} + \frac{1}{p'} = 1$ (also for $p=1,\ p' = +\infty$).
\end{corollary}
\begin{proof}\
\begin{itemize}
    \item Easy part:\\
    $g \in \Ls^{p'}$ induces a bounded linear functional on $\Ls^p$ (\hyperref[RRT]{Riesz Representation Theorem}) by setting 
    \begin{equation*}
        F(f) = \int_\Omega fg d\mu.
    \end{equation*} 
    Use H\"older inequality, we get
    \begin{equation*}
        \abs{F(f)} \leq \norm{f}_p \norm{g}_{p'},\ \norm{F} = \norm{g}_{p'}    
    \end{equation*} 
    by choosing $f = \abs{g}^{p' - 1} \frac{\bar{g}}{\abs{g}}$, since 
    \begin{equation*}
        F(f) = \int_\Omega \abs{g}^{p'} d\mu = \norm{g}_{p'}^{p'}
    \end{equation*}
    From $\frac{1}{p} + \frac{1}{p'} = 1$ \imply $p' - 1 = \frac{p'}{p}$, we have 
    \begin{equation*}
        \norm{f}_{p}^p = \int \abs{f}^p d\mu = \int \abs{g}^{p'} d\mu = \norm{g}_{p'}^{p'}\ \Rightarrow\ \norm{f}_p = \norm{g}_{p'}^{p'/p} = \norm{g}_{p'}^{p' - 1}
    \end{equation*}
    which implies
    \begin{equation*}
      \norm{g}_{p'}^{p'} = \norm{g}_{p'} \norm{f}_p\ \Rightarrow\ \norm{F} = \norm{g}_{p'}.  
    \end{equation*}
    \begin{remark}
    Note $\sup_{\norm{f} = 1} \abs{F(f)}$ is attained for $1 < p < \infty$ with $f = g^{p'-1} \frac{\bar{g}}{\abs{g}}$, which shows $F$ is bounded and 
    $$\Ls^{p'}(\Omega,\Sigma, \mu) \subset \text{ dual of } \Ls^p(\Omega, \Sigma, \mu).$$ The supremum is also attained when $p = \infty$, $g \in \Ls^1$ and let $f = \frac{\bar{g}}{\abs{g}}$. The supremum is not attained when $p = 1$, $g \in \Ls^\infty$. Let 
    \begin{equation*}
        F(f) = \int fg d\mu.
    \end{equation*}
    If $g(\cdot)$ is continuous on $\R$ with unique maximum, then $\sup_{\norm{f} = 1} \abs{F(f)}$ is not attained.  
    \end{remark}
    \begin{proof}[Summary]\ \\
    If $1 \leq p \leq \infty$, $\Ls^{p'}$ is contained in the dual of $\Ls^p$.\\ 
    If $1 < p \leq \infty$, then $\sup_{\norm{f}_p = 1} \abs{F(f)}$ is attained.\\ 
    For $p = 1$, the supremum is not attained.
    \end{proof}
    \item For $E = \Ls^p(\Omega,\Sigma,\mu)$ with $1 \leq p < \infty$, and $F \in E^*$, we consider finite measure space $\mu(\Omega) < \infty$. Define measure $\nu$ on $\Sigma$ by
    \begin{equation*}
        \nu(A) = F(\1_A) = \int_A h \dr \mu,\ A \in \Sigma,
    \end{equation*}
    where $h \in \Ls_0(\mu)$ by the \hyperref[RRT]{Riesz Representation Theorem}, and we have
    \begin{center}
        $\mu(A) = 0$ \imply $\nu(A) = 0$ \imply $\nu \ll \mu$.
    \end{center}
    By \hyperref[R-N thm]{Radon-Nikodym Theorem}, we know
    \begin{equation*}
        \nu(A) = \int_A g d\mu
    \end{equation*}
    for some $g \in \Ls^1$ s.t. $g = \frac{d\nu}{d\mu}$. It suffices to show $g \in \Ls^{p'}$. Note that $p' > 1$, we have
    \begin{equation*}
        F(f) = \int_{\Omega} fg d\mu
    \end{equation*}
    for simple functions $f$. Can assume $g \geq 0$ since $\nu = \nu^+ - \nu^-$ is a signed measure, and we can decompose it and apply R-N Theorem respectively. Then we can use MCT and $f = f^+ - f^-$ to extend the result s.t.
    \begin{equation*}
            F(f) = \int_{\Omega} fg d\mu
    \end{equation*}
    holds for all $f \in \Ls^p$. Can set $f = g^{p' - 1}$, and use 
    \begin{equation}
        \abs{F(f)} \leq \norm{F} \norm{f}_p\label{eq:4.3}
    \end{equation}
    where $\frac{1}{p} + \frac{1}{p'} = 1$ \imply $p' - 1 = \frac{p'}{p}$. And \hyperref[eq:4.3]{Equation 4.3} implies
    \begin{equation*}
        \int g^{p'} d\mu \leq \norm{F} (\int g^{p'} d\mu)^{\frac{1}{p}},
    \end{equation*}
    and thus
    \begin{align*}
        \norm{g}_{p'}^{p'} &\leq \norm{F} \norm{g}_{p'}^{\frac{p'}{p}}\\
        &= \norm{F} \norm{g}_{p'}^{p' - 1}
    \end{align*}
    which concludes $\norm{g}_{p'} \leq \norm{F} < +\infty$. Complete the proof.
\end{itemize}
\end{proof}

\begin{remark}\ \\
$\Ls^1$ is a subset of the dual of $\Ls^\infty$, but not equal to it. Note if $F: \Ls^\infty (\mu) \to \C$ is a bounded linear functional, then if $\Omega = \mathrm{K} = \text{ compact Hausdorff space } F$ induces a bounded linear functional on $C(K) = \text{ space of continuous functions on } K$, we have $C(k) \subset \Ls^\infty (K, \Sigma, \mu)$, where $\Sigma = \Bs(K)$. \textcolor{red}{The counterexample is a little complex, check later}
\end{remark}

\begin{theorem}[Riesz Representation Theorem, v.2]\label{RRT2}\ \\
Let $E = C(K)$ be the space of continuous functions on compact Hausdorff space $K$. Then,
\begin{enumerate}[label = (\alph*)]
    \item For every Borel regular signed measure $\mu$ on $K$, the functional $F(f) = \int_K f d\mu$ is a bounded linear functional on $K$.
    \item Every bounded linear functional on $C(K)$ can be expressed in (a) for some measure $\mu$, and $\norm{F} = \abs{\mu} (K)$, i.e. $\text{Tv}(K)$, where $\text{Tv}(\cdot)$ denotes the total variation. 
\end{enumerate}
\end{theorem}


\vspace{12pt}
\subsection{Hahn-Banach Theorem}
\begin{theorem}[Hahn-Banach Theorem]\label{HB thm}\ \\
Let $E_0$ be a subspace of a normed space, then every bounded linear functional $f_0: E_0 \to \R \text{ or } \C$ has an extension $f: E \to \R \text{ or } \C$ s.t. $\norm{f} = \norm{f_0}$.
\end{theorem}
\begin{proof}\ \\
Assume $E$ is separable, otherwise we need "transfinite induction". 
\begin{remark}
Separability allows us to extend $f_0$ 1 dimension at a time.
\end{remark}
Let $\{x_n: n = 1,2,\dots\}$ have the property that its span is dense in $E$. We can do induction:
\begin{equation*}
    E_0 \to E_0 + \{x_1\} \to E_0 + \{x_1,x_2\} \to \dots \to E_0 + \vspan{x_1,\dots,x_n},\ \norm{f} = \norm{f_0}.
\end{equation*}
This final space is dense in $E$ and so can extend $f$. Thus it suffices to prove the extension by 1 dimension s.t. $E \to E_0 + \{x_1\}$.

\np Note the extension is determined by a single number $\gamma = f(x_1)$. First we consider linear functional $f_0: E_0 \to \R$ and extend to $f: E_0 + \{x_1\} \to \R$ with $\norm{f} = \norm{f_0}$, which is equivalent to (WLOG, assuming $\norm{f} = \norm{f_0} = 1$)
\begin{equation*}
    \abs{f_0(x_0) + \lambda \gamma} \leq \norm{x_0 + \lambda x_1},\ \forall\ x_0 \in E_0,\ \lambda \in \R.
\end{equation*}
Divide inequality by $\lambda \neq 0$, it suffices to find $\gamma$ s.t.
\begin{equation*}
    \abs{f_0(x_0) + \gamma} \leq \norm{x_0 + x_1},\ \forall\ x_0 \in E_0.
\end{equation*}
Note we have $f_0$ is a real functional, it is equivalent to
\begin{equation*}
    - \norm{x_0 + x_1} \leq f_0(x_0) + \gamma \leq \norm{x_0 + x_1},\ \forall\ x_0 \in A.
\end{equation*}
Such a $\gamma$ exists provided
\begin{equation*}
    \norm{x_0 + x_1} - f_0(x_0) \geq - \norm{x_0'+ x_1} - f(x_0'),\ \forall\ x_0,x_0' \in E_0,
\end{equation*}
which is equivalent to (assuming $\norm{f_0} = 1$)
\begin{align*}
    f_0(x_0 - x_0') &\leq \norm{x_0 + x_1} + \norm{x_0' + x_1},\ \forall\ x_0,x_0' \in E_0\\
    f_0(x_0 - x_0') &\leq \norm{x_0 + x_1} + \norm{-x_0' - x_1},\ \forall\ x_0,x_0' \in E_0
\end{align*}
which can be given by
\begin{equation*}
    f_0(x_0 - x_0') \leq \norm{x_0 - x_0'} \underbrace{\leq}_{\text{triangle inequality}} \norm{x_0 + x_1} + \norm{-x_1 - x_0'}
\end{equation*}

\np \underline{Extension to linear functionals over complexs}:\\
Let $f: E \to \C$ be a linear functional over $\C$, set $g(x) = \Re{f(x)}$, then $g: E \to \R$ is a real linear functional, and $f(x) = g(x) - i g(ix),\ x \in E$. Note
\begin{align*}
    f(ix) &= i f(x)\\
    g(ix) &= - \Im{f(x)}
\end{align*}
Conversely, if $g: E \to \R$ is a real linear functional on Banach space (not necessarily need Banach) $E$ over $\C$, then
\begin{equation*}
    f : E \to \C
\end{equation*}
defined by
\begin{equation*}
    f(x) = g(x) - i g(ix),\ x \in E
\end{equation*}
is a complex linear functional on $E$.

\np To extend $f_0: E_0 \to \C$, we extend $g_0 = \Re{f_0}$ 2 real dimensions s.t.
\begin{equation*}
    E_0 \to E_0 + x_1 \to E_0 + x_1 + i x_1
\end{equation*}
and define $f(\cdot) = g(\cdot) - i g(i \cdot)$.

\np To show $\norm{f} = \norm{f_0}$, we use that fact that for $x \in E_0 + \{\lambda x_1:\ \lambda \in \C\}$, we have
\begin{equation*}
    e^{i \theta} f(x) = f(x e^{i\theta}),\ \theta \in \R.
\end{equation*}
We can choose $\theta$ s.t. $f(x e^{i\theta}) = \Re{g(x e^{i \theta})}$. Now we already have
\begin{equation*}
    \abs{g(x e^{i \theta})} \leq \norm{f_0} \norm{x e^{i \theta}}
\end{equation*}
which implies 
\begin{equation*}
    \abs{f(x)} \leq \norm{f_0} \norm{x},\ \forall\ x \in E_0 + \{\lambda x_1:\ \lambda \in \C\}.
\end{equation*}
\end{proof}







\vspace{12pt}
\subsubsection{Implications of the Hahn-Banach Theorem}
\begin{proposition}[Supporting Hyperplane Theorem]\label{SHT}\ \\
Let $E$ be a normed space, for every $x \in E$, $\exists\ f \in E^*$ s.t. 
\begin{equation*}
    \norm{f} = 1,\ f(x) = \norm{x},
\end{equation*}
i.e. $\sup_{\norm{y}=1} \abs{f(y)}$ is attained at $y = x$. 
\end{proposition}
\begin{proof}\ \\
Consider 1-dimensional space $E_0 = \vspan{x} = \{tx: t \in \R \text{ or } \C\}$. Define $f_0: E_0^* \to \R \text{ or } \C$ s.t.
\begin{equation*}
    f_0(tx) = t \norm{x}.
\end{equation*}
We have $\norm{f_0} = 1$, and by \hyperref[HB thm]{Hahn-Banach Theorem}, $\exists\ f \in E^*$ s.t. $\ f(x) = \norm{x},\ \norm{f} = 1$.
\end{proof}
\begin{remark}[Geometric Interpretation]\ \\
Suppose $B = \text{ unit ball } \{x \in E: \norm{x} \leq 1\}$. Choose $x_0 \in \partial B,\ \norm{x_0} = 1$. $\exists\ f \in E^*$ s.t. $\norm{f} = 1,\ f(x) = \norm{x}$. Let 
\begin{equation*}
    H = \ker{f} + x_0,
\end{equation*}
where $\ker{f} = \{x: f(x) = 1\}$. $H$ intersects $E$ into two disjoint subsets, and $B$ lies entirely in one of them, i.e. for all $x \in B$ we have
\begin{center}
    $\norm{x} < 1$ \imply $\abs{f(x)} \leq \norm{f} < 1$ \imply f(x) < 1,
\end{center}
and we also have $E = \{x: f(x) < 1\} \kup H \kup \{x: f(x) < 1\}$.
\end{remark}
\begin{remark}\ \\
Tangent hyperplane $H$ is not necessarily unique. We can extend this to prove the existence of supporting hyperplane for more general convex sets.
\end{remark}

\vspace{12pt}
\subsection{Second Duality}

\begin{definition}\ \\
    Let $E$ be a Banach space, the \underline{duality of $E$} is the space of all bounded linear functionals on $E$. We have for $f \in E^*,\ \norm{f} = \sup_{\norm{x} = 1} \abs{f(x)}$, and $E^*$ is a Banach space.  
\end{definition}

\np Let $E^*$ be the dual of $E$, and $E^{**}$ be the dual of $E^*$. $\exists$ natural embedding $E \to E^{**}$, $x \to x^{**} \in E^{**}$ s.t.
\begin{equation*}
    x^{**}(f) = f(x),\ f \in E^*.
\end{equation*}
We have
\begin{equation*}
    \norm{x^{**}} = \sup_{\norm{f} = 1,\ f \in E^*} \abs{x^{**} (f)} = \sup_{f \in E^*,\ \norm{f} = 1} \abs{f(x)} \leq \norm{x},
\end{equation*}
which implies
\begin{equation*}
    \norm{x^{**}} \leq \norm{x},\ x \in E.
\end{equation*}
By Hahn-Banach Theorem, we have $\norm{x^{**}} = \norm{x}$. Recall 
\begin{equation*}
    f_0: f_0(tx) = t \norm{x},\ t \in R \text{ or } \C,\ \norm{f_0} = 1.
\end{equation*}
Extend $f_0$ to a functional $f_x: E \to \R \text{ or } \C$ s.t. $\norm{f_x} = 1,\ f_x(x) = \norm{x}$ \imply $x^{**}(f_x) = \norm{x}$ \imply $\norm{x^{**}} \geq \norm{x}$.

\np Conclude the embedding $E \to E^{**}$ is an isometry if the mapping is onto, say $E$ is a reference space.

\vspace{6pt}
\begin{theorem}[Second Dual Space]\ \\
Let $E$ be a normed space, then $E$ can be considered as a linear subspace of $E^{**}$. For this, a vector $x \in E$ is considered as a bounded linear functional on $E^*$ via the action:
\begin{equation*}
    x: f \to f(x),\ f \in E^*.
\end{equation*}
\end{theorem}


\begin{examples}\ 
\begin{enumerate}[label = (\alph*)]
    \item Hilbert spaces.
    \item $E = \Ls^p$ spaces for $1 < p < \infty$, we have
    \begin{equation*}
        E^* = \Ls^{p'},\ \frac{1}{p} + \frac{1}{p'} = 1,\ 1 < p' < \infty
    \end{equation*}
    and $E^{**} = \Ls^p$.
\end{enumerate}
\end{examples}


\begin{definition}[Reflexivity]\ \\
    A normed vector space $E$ is called \textit{reflexive} if $E^{**} = E$ under the natural embedding.
\end{definition}

\begin{proposition}\ \\
If $E$ is a reflexive normed space, then every functional $f \in E^*$ attains its maximum on $E$.
\end{proposition}
\begin{proof}\ \\
E is reflexive and $f \in E^*$, then $\exists\ x_f \in E^{**} = E$ with $\norm{x_f} = 1$ and $\norm{f} = f(x_f)$, i.e. $\sup_{\norm{x} = 1} \abs{f(x)}$ is achieved at $x = x_f$. This follows from \hyperref[SHT]{Supporting Hyperplane Theorem}.
\end{proof}

\np To show a normed space $E$ is not reflexive, it suffices to find $f \in E^*$ s.t. $\sup_{\norm{x} = 1} \abs{f(x)}$ is not attained. 

\begin{example}\ \\
Let $E = C([0,1]) = \text{ space of continuous functions } g: [0,1] \to \C$, and $\norm{g} = \sup_{0 \leq t\leq 1} \abs{g(t)}$. For $f \in E^*$, we define
\begin{equation*}
    f(g) = \int_0^1 h(x) g(x) dx,
\end{equation*}
where $h(x) = -1$ for $0  < x < \frac{1}{2}$, and $h(x) = 1$ for $\frac{1}{2} < x < 1$. Then
\begin{equation*}
    \norm{f} = 1 = \sup_{\norm{g} = 1} \abs{f(g)},
\end{equation*}
but sup is not attained.
\end{example}


\vspace{12pt}
\subsection{Separating Hyperplane Theorem}
\begin{remark}
Separating Hyperplane Theorem is an extension of \hyperref[SHT]{Supporting Hyperplane Theorem}, generalizing to arbitrary convex sets.
\end{remark}

\subsubsection{Sublinear Functionals}
\begin{definition}\ \\
    Let $E$ be a linear space, a function
    \begin{equation*}
        \norm{\cdot} : E \to [0, \infty)
    \end{equation*}
    is \textit{sublinear} if it satisfies
    \begin{enumerate}[label = (\alph*)]
        \item $\norm{\lambda x} = \lambda \norm{x},\ \lambda \in \R,\ \lambda > 0,\ x \in E$.
        \item $\norm{x + y} \leq \norm{x} + \norm{y},\ x,y \in E$.
    \end{enumerate}
\end{definition}
\begin{remark}\ \\
Note the difference of the definition of sublinear functions and the norm. For a sublinear function $\norm{\cdot}$, it requires two additional properties to become a norm:
\begin{itemize}
    \item $\norm{-x} = \norm{x},\ \forall\ x\in E$.
    \item $\norm{x} = 0 \Rightarrow x = 0$.
\end{itemize}
\end{remark}

\vspace{6pt}
\begin{theorem}[Extension of Hahn-Banach Theorem]\ \\
Let $E_0$ be a subspace of a linear vector space over $\R$, and $\norm{\cdot}$ be a sublinear functional on $E$ and $f_0: E_0 \to \R$ a linear functional on $E_0$ satisfying
\begin{equation*}
    f_0(x) \leq \norm{x},\ x \in E_0.
\end{equation*}
Then $f_0$ admits an extension $f$ on $E$ s.t.
\begin{equation*}
    f(x) \leq \norm{x},\ x \in E.
\end{equation*}
\end{theorem}
\begin{proof}\ \\
Same argument as in the proof of \hyperref[HB thm]{Hahn Banache Theorem}.
\end{proof}
\begin{remark}\ \\
There is no guarantee that if $E$ is a Banach space, then $f(\cdot)$ is a bounded linear functional.
\end{remark}

\subsubsection{Geometric Properties of Sublinear Functionals}
\begin{definition}\ \\
    A subset $K$ of a linear vector space $E$ is \textit{absorbing} if
    \begin{equation*}
        E = \bigcup_{t \geq 0} t K,
    \end{equation*}
    where $tK = \{tk: k \in K\}$.
\end{definition}

\begin{proposition}[Minkowski Functional]\ \\
Let $K \subset E$ be an absorbing convex subset of a linear vector space $E$ s.t. $0 \in K$. Then the \textit{Minkowski functional}
\begin{equation*}
    \norm{x}_K = \inf\{t > 0: x \in tK\}
\end{equation*}
is a sublinear functional on $E$. Conversely, let $\norm{\cdot}$ be a sublinear functional on $E$. Then the \textit{sub-level} set
\begin{equation*}
    K = \{x \in E:\ \norm{x} \leq 1\}
\end{equation*}
is an absorbing convex set and $0 \in K$.
\end{proposition}
\begin{proof}\ \\
\begin{itemize}
    \item $\Rightarrow:$\\
    Let $K$ be an absorbing convex set with $0 \in K$, the main observation is that since $0 \in K$, and $K$ is convex, we have
    \begin{equation*}
        x \in K \Rightarrow tx \in K,\ 0 \leq t < 1.
    \end{equation*}
    WTS $\norm{\lambda x} = \lambda \norm{x},\ x \in E,\ \lambda > 0$. Note
    \begin{align*}
        \norm{\lambda x} &= \inf\{t > 0:\ \lambda x \in tK\}\\
        &= \inf\{t > 0:\ x \in \frac{t}{\lambda} K\}\\
        &= \lambda \inf\{s > 0:\ x \in sK\}\\
        &= \lambda \norm{x}.
    \end{align*}
    It remains to prove the triangle inequality. For $x\in tK,\ y \in sK$, we have 
    \begin{equation*}
        x = tk_1,\ y = sk_2,\ k_1,k_2 \in K,
    \end{equation*}
which implies
\begin{align*}
    x + y &= (t + s) [\frac{t}{t+s} k_1 + \frac{s}{t+s} k_2]\\
    &= (t+s) k
\end{align*}
for some $k \in K$, since $K$ is convex. Therefore, we have
\begin{equation*}
    x + y \in (t+s)K \Rightarrow \norm{x + y} \leq \norm{x} + \norm{y},
\end{equation*}
since $\{t: x \in tK\}$ is an interval $(\alpha(x), \infty)$.
\item $\Leftarrow:$\\
$0 \in K$ since $\norm{0} = 0$, and the convexity comes from triangle inequality.\\
\begin{remark}
If $K \neq -K$, then $\exists\ x \in E$ with $\norm{x} \neq \norm{-x}$ if $K = E$, then $\norm{\cdot} = 0$.
\end{remark}
\end{itemize}
\end{proof}


\subsubsection{Seperating Hyperplane}

\begin{theorem}[Separation of a Point from a Convex Set]\ \\
Let $K$ be an open convex subset of a normed space $E$ and $x_0 \neq K$. Then exists a continuous linear functional
\begin{equation*}
    f: E \to \R,\ f \not\equiv 0,
\end{equation*}
and $f(x) < f(x_0),\ x \in K$.
\end{theorem}
\begin{proof}\ \\
By translation can assume WLOG $0 \in K$. Since $K$ is open, it is absorbing. Let $\norm{\cdot}_K$ be Minkowski functional, then
\begin{equation*}
    \norm{x}_K \leq \frac{1}{r} \norm{x},\ x \in E
\end{equation*}
if $B(0,r) \subset K$. Proceed as in separation theorem for a unit ball. Define $f_0$ on $\vspan{x_0}$ by
\begin{equation*}
    f_0(tx_0) = t\norm{x_0}_K,\ t \in \R,
\end{equation*}
then if $E_0 = \{\lambda x_0:\ \lambda \in \R\}$, we have
\begin{equation*}
    f_0(x) \leq \norm{x}_K,\ x \in E_0,
\end{equation*}
since \begin{equation*}
    \norm{tx_0}_K = t \norm{x_0}_K,\ t \geq 0,
\end{equation*}
while for $t \leq 0$,
\begin{equation*}
    f_0(tx_0) = t f_0(x_0) \leq 0 \leq  \norm{t x_0}_K.
\end{equation*}

\np \hyperref[HB thm]{Hahn Banach Theorem} implies that we can extend $f_0$ to $f: E \to \R$ s.t.
\begin{equation*}
    f(x) \leq \norm{x}_K \leq \frac{1}{r} \norm{x},\ x \in E,
\end{equation*}
and thus $f$ is a bounded linear functional.\\
For separation, we have
\begin{equation*}
    f(x) \leq \norm{x}_K \leq 1 \leq \norm{x_0}_K = f_0(x_0) = f(x_0),\ x \in K,
\end{equation*}
which implies $f(x) \leq f(x_0),\ x \in K$, and since $K$ is open,
\begin{equation*}
    x + tv \in K
\end{equation*}
for some $t > 0$, $\forall\ v$ s.t. $\norm{v} = 1$. Hence,
\begin{equation*}
    f(x + tv) \leq f(x_0),\ t = t_x > 0,\ \norm{x} = 1,
\end{equation*}
and thus 
\begin{equation*}
    f(x) + t \sup_{\norm{v} = 1} f(v) \leq f(x_0).
\end{equation*}
Note that $\sup_{\norm{v} = 1} f(v) > 0$, we have
\begin{equation*}
    f(x) < f(x_0).
\end{equation*}
\end{proof}


\vspace{3pt}
\begin{theorem}[Separation of Convex Sets]\label{SepConvSets}\ \\
Let $A,B$ be disjoint convex subsets of a normed space $E$,
\begin{enumerate}[label = (\alph*)]
    \item If $A$ is open, then $\exists$ a bounded linear functional
    \begin{equation*}
        f: E \to \R
    \end{equation*}
    s.t. $f(a) < f(b),\ \forall\ a \in A,\ b \in B$.
    \item If $A$ is closed and $B$ is compact, then
    \begin{equation*}
        \sup_{a \in A} f(a) < \inf_{b \in B} f(b)
    \end{equation*}
\end{enumerate}
\end{theorem}
\begin{proof}\
\begin{enumerate}[label = (\alph*)]
    \item Let 
    \begin{equation*}
        K = A -B = \{a - b:\ a \in A,\ b \in B\},
    \end{equation*}
    then $K$ is open, convex and $0 \not\in K$. By the previous theorem, $\exists\ f \in E$ s.t.
    \begin{equation*}
        f(a - b) < f(0) = 0,\ \forall\ a \in A,\ b \in B.
    \end{equation*}
    Therefore,
    \begin{equation*}
        f(a) < f(b),\ a \in A,\ b \in B.
    \end{equation*}
    \item Suppose $A$ closed and $B$ compact, we have
    \begin{equation*}
        d(A,B) := \inf\{\norm{x - y}:\ x \in A,\ y \in B\} = \delta > 0.
    \end{equation*}
    Define $A_\delta = \{x \in E:\ d(x,A) < \delta\}$, we have $A_\delta$ is open, and take
    \begin{equation*}
        \ep = \frac{\delta}{2},
    \end{equation*}
    we have $A_\ep \cap B = 0$. By part (a), we have $\exists\ f \in E^*$ s.t.
    \begin{equation*}
        f(x) < f(y),\ \forall\ x \in A,\ y \in B.
    \end{equation*}
    For $a \in A$, we have $a + \frac{\delta}{2} v \in A_\delta$ if $\norm{v} = 1$. Thus
    \begin{equation*}
        f(a + \frac{\delta}{2} v) = f(a) + \frac{\delta}{2} f(v) < f(b),\ b \in B.
    \end{equation*}
    Take sup over $\norm{v} = 1$, we have
    \begin{equation*}
        \sup_{\norm{v} = 1} \abs{f(v)} = \ep > 0,
    \end{equation*}
    which implies 
    \begin{equation*}
        f(a) < f(b) - \ep,\ a \in A,\ b \in B,
    \end{equation*}
    which further implies
    \begin{equation*}
        \sup_{a \in A} f(a) < \inf_{ b \in B} f(b).
    \end{equation*}
\end{enumerate}
\end{proof}

\begin{corollary}\ \\
Let $K \subset E$ be a closed convex set, then $K$ is the intersection of all half spaces containing $K$, where half space $H = \{x \in E:\ f(x) \leq \lambda,\ f \in E^*\}$.
\end{corollary}
\begin{proof}\ \\
If $x \notin K$, then $\exists\ f \in E^*$ s.t.
\begin{equation*}
    \sup_{k \in K} f(k) < f(x_0).
\end{equation*}
Can choose $\lambda$ s.t. $\sup_{k \in K} f(k) \leq \lambda \leq f(x_0)$.
\end{proof}













